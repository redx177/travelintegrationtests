% !TeX encoding=utf8
% !TeX spellcheck = de_CH_frami

\subsection*{Abstrakt}
In der Softwareentwicklung wird traditionell nach dem Wasserfall-Prinzip\footcite{Wasserfallmodell} vorgegangen. Viele Teams wird dieses alte Vorgehensmodel durch neue, agile Methoden abgelöst. In kurzen abständen (3-4 Wochen) wird dem Klient eine lauffähige Version vorgestellt. Kunden werden dabei stärker in den Entwicklungsprozess eingebunden, wodurch neue Probleme entstehen. In jedem Zyklus ist eine Spezifikations-, eine Umsetzungs- und eine Test-Phase enthalten. Bei umfangreicher Software kann die letzterer Abschnitt sehr umfangreich werden und wird deshalb gerne vernachlässigt.

Um diesem Problem entgegenzutreten kann das Testen der Software Automatisiert werden. Diesem Thema nimmt sich diese Arbeit an. Sie versucht am Beispiel der travel.ch aufzueigen, wie ein Automated Testing geplannt und umgesetzt werden kann, und wann sich dieses lohnt und wann nicht.

%
\mbox{}\\[0.5\baselineskip]\noindent
\textbf{Schlagwörter:} 
Integration Tests, Selenium, Testing
