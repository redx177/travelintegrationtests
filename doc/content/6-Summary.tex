
\chapter{Zusammenfassung \& Fazit}
In dieser Arbeit wurden Selenium Tests für die Webseite travel.ch konzipiert und umgesetzt. 

Als erster Schritt wurde eine Recherche betrieben, dessen Ziel es war alle benötigten Informationen für die Umsetzung aufzutreiben. Die wichtigsten Ergebnisse aus dieser Phase ist der Entscheid, auf Grundlage eines Prototypen, Service-Provider einzusetzen anstelle die Tests intern auszuführen (siehe \cref{sec:Recherche:loesungswege} \nameref{sec:Recherche:loesungswege}). 

Danach wurden die Befunde aus der Recherche Analysiert und die Anforderungen niedergeschrieben. Als Ergebnis kristallisierten sich folgende drei Anforderungen heraus:
\begin{itemize}
\item Funktionalität
\item Browserabdeckung
\item Kosten \& Zeit
\end{itemize}
Ein Grossteil der Arbeit war die Funktionalität zu definieren, da diese aus der bestehenden Webseite durch reverse Engineering abgeleitet werden musste (siehe \cref{sec:analyse:Funktionalität} \nameref{sec:analyse:Funktionalität}). Ein weiteres Problem bestand darin, dass die "`Kosten \& Zeit"' stiegen, je höher die Funktionalität und die Browserabdeckung gewählt wurden. Es wurde deshalb ein Kompromiss mit der travelwindow AG eingegangen, dass zwei Browser und jede Funktionalität mit einer Priorität höher als 8 umgesetzt werden sollen (siehe  \cref{sec:analyse:testabdeckung} \nameref{sec:analyse:testabdeckung}).

Mit der gewünschten Funktionlität und der Browserabdeckung konnte mit dem Konzept begonnen werden. Zuerst wurden die Tests in die drei Bereiche "`Happy Path"', "`Smoke Tests"' und "`Main Tests"' aufgeteilt. Danach die Eingabeparameter definiert und schlussendlich die zu implementierenden Tests spezifiziert.

Schlussendlich konnte mit der Entwicklung begonnen werden. Es wurden die beiden Design Patterns "`Page Object"' und "`Page Factory"' eingesetzt, welche die Lesbarkeit der Tests markant verbesserte (siehe \cref{sec:umsetzung:patterns} \nameref{sec:umsetzung:patterns}). Die Umsetzung schritt mit den getätigten Vorbereitungen rasant voran. Das erste Problem trat auf, als bei den Tests auf die Webseite gewartet werden musste, da Bereiche dynamisch nachgeladen werden. Die Problematik konnte jedoch mit angepassten Timings behoben werden (siehe \cref{sec:umsetzung:probleme:timing} \nameref{sec:umsetzung:probleme:timing}). Wesentlich mehr Schwierigkeiten generierten die Komplexität der Tests (siehe \cref{sec:umsetzung:probleme:komplexität} \nameref{sec:umsetzung:probleme:komplexität}). Diese Erschwernis konnte noch nicht abschliessend gelöst werden.

Abschliessend lässt sich sagen, dass Selenium Tests als mächtiges Hilfsmittel für die Qualitätssteigerung in der Softwareentwicklung eingesetzt werden kann. Für die travel.ch Webseite sind sie jedoch nur teilweise geeignet, da der ganze Prozess von der Suche bis zur definitiven Buchung sehr lang ist. Wenn dabei irgendwo ein Fehler auftritt schlägt der ganze Test fehl. Diese Problematik wird verstärkt, dass es keine Stubs gibt um die Willkür der Suchresultate Herr zu werden (siehe \cref{sec:umsetzung:probleme:komplexität} \nameref{sec:umsetzung:probleme:komplexität}) und die Infrastruktur sehr instabil ist (siehe \cref{sec:umsetzung:probleme:stabilität} \nameref{sec:umsetzung:probleme:stabilität}).