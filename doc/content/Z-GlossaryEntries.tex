% !TeX encoding=utf8
% !TeX spellcheck = de_CH_frami

%%% --- Acronym definitions
\IfDefined{newacronym}{%
	\newacronym{svn}{SVN}{Subversion}
}

%%% --- Symbol list entries

%\newglossaryentry{symb:Pi}{%
%  name=$\pi$,%
%  description={mathematical constant},%
%  sort=symbolpi, type=symbolslist%
%}


%%% --- Glossary entries
\newglossaryentry{glos:virtualMachine}{name=Virtual Machine,
	description={Eine virtuelle Machine bildet einen Rechner einer real existierender Hardware in einer virtuellen Umgebung nach. Dabei handelt es sich um ein vollständiges Betriebssystem, welches sich exakt so verhält, als sei es direkt auf der Hardware installiert.}
}
\newglossaryentry{glos:session}{name=Session,
	description={Eine Session ist eine stehende Verbindung eines Clients mit einem Server. Da das Internet auf dem zustandslosen Protokoll HTTP (Hypertext Transfer Protocol) basiert muss die Sitzung auf der Anwendungsschicht\footcite{OSI-Modell__Wikipedia_2015-07-30} etabliert werden. Dazu wird bei der ersten Kommunikation eine eindeutige Kennung vom Server an den Client übermittelt, welche dieser bei jedem Verbindungsaufbau zurück übermittelt.}
}
\newglossaryentry{glos:httpcookie}{name=HTTP-Cookie,
	description={Ein (HTTP-)Cookie ist eine Information, die der Server über den Browser auf dem Computer des Betrachters abspeichert. Die Cookie-Daten werden vom Browser bei jeder Anfrage an den Server zurück übermittelt.}
}
\newglossaryentry{glos:failureSafetyLabel}{name=Ausfallsicherheit,
	description={Mit der Ausfallsicherheit wird die minimale zeitliche Erreichbarkeit (resp. maximale Ausfallzeit) eines Systems angegeben. Ist diese Ausfallzeit sehr gering spricht man von Hochverfügbarkeit (High Availability), dazu ist mindestens eine Verfügbarkeit von 99.9 \% nötig.
	Die Verfügbarkeit berechnet man wie folgt:
	\begin{gather*}
		\text{Verfügbarkeit} = (1- \frac{\text{Ausfallzeit}}{\text{Periode}}) * 100\\
		\text{Ausfallzeit} = (1 - \frac{\text{Verfügbarkeit}}{100}) * \text{Periode}
	\end{gather*}}
}
\newglossaryentry{glos:thinClientLabel}{
  name=Thin Client,
  plural={Thin Clients},
  description={Ein Thin Client ist ein günstiger, rechen-schwacher Computer. Er wird dazu verwendet, um Arbeiten zu erledigen die auf einem rechen-starken Server statt finden. Ein Thin Client übernimmt hauptsächlich die Bereitstellung von }
}
\newglossaryentry{glos:onPremiseLabel}{name=on-premise,
	description={On-premise Software bezeichnet jegliche Ausführung von Software die vorwiegend auf dem Endgerät selbst läuft. Alternativ existiert das \Gls{glos:cloudLabel} Modell, bei dem der Grossteil der Software auf einem Server ausgeführt wird.}
}
\newglossaryentry{glos:cloudLabel}{
	name=Cloud,
	description={Der Begriff \textit{Cloud} ist im definiert.}
}



% use it with \gls{glos:DVD}
% use plural with \glspl{thinClientLabel}
