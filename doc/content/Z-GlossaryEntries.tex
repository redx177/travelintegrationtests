% !TeX encoding=utf8
% !TeX spellcheck = de_CH_frami

%%% --- Acronym definitions
\IfDefined{newacronym}{%
	\newacronym{svn}{SVN}{Subversion}
}

%%% --- Symbol list entries

%\newglossaryentry{symb:Pi}{%
%  name=$\pi$,%
%  description={mathematical constant},%
%  sort=symbolpi, type=symbolslist%
%}


%%% --- Glossary entries
\newglossaryentry{glos:failureSafetyLabel}{name=Ausfallsicherheit,
	description={Mit der Ausfallsicherheit wird die minimale zeitliche Erreichbarkeit (resp. maximale Ausfallzeit) eines Systems angegeben. Ist diese Ausfallzeit sehr gering spricht man von Hochverfügbarkeit (High Availability), dazu ist mindestens eine Verfügbarkeit von 99.9 \% nötig.
	Die Verfügbarkeit berechnet man wie folgt:
	\begin{gather*}
		\text{Verfügbarkeit} = (1- \frac{\text{Ausfallzeit}}{\text{Periode}}) * 100\\
		\text{Ausfallzeit} = (1 - \frac{\text{Verfügbarkeit}}{100}) * \text{Periode}
	\end{gather*}}
}
\newglossaryentry{glos:thinClientLabel}{
  name=Thin Client,
  plural={Thin Clients},
  description={Ein Thin Client ist ein günstiger, rechen-schwacher Computer. Er wird dazu verwendet, um Arbeiten zu erledigen die auf einem rechen-starken Server statt finden. Ein Thin Client übernimmt hauptsächlich die Bereitstellung von }
}
\newglossaryentry{glos:onPremiseLabel}{name=on-premise,
	description={On-premise Software bezeichnet jegliche Ausführung von Software die vorwiegend auf dem Endgerät selbst läuft. Alternativ existiert das \Gls{glos:cloudLabel} Modell, bei dem der Grossteil der Software auf einem Server ausgeführt wird.}
}
\newglossaryentry{glos:cloudLabel}{
	name=Cloud,
	description={Der Begriff \textit{Cloud} ist im \cref{sec:cloud:definition} definiert.}
}



% use it with \gls{glos:DVD}
% use plural with \glspl{thinClientLabel}
