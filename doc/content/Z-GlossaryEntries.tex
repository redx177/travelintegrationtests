% !TeX encoding=utf8
% !TeX spellcheck = de_CH_frami

%%% --- Acronym definitions
\IfDefined{newacronym}{%
	\newacronym{svn}{SVN}{Subversion}
	\newacronym{dom}{DOM}{Document Object Model}
	\newacronym{url}{URL}{Uniform Resource Locator}
}

%%% --- Symbol list entries

%\newglossaryentry{symb:Pi}{%
%  name=$\pi$,%
%  description={mathematical constant},%
%  sort=symbolpi, type=symbolslist%
%}


%%% --- Glossary entries
\newglossaryentry{glos:virtualMachine}{name=Virtual Machine,
	description={Eine virtuelle Machine bildet einen Rechner einer real existierender Hardware in einer virtuellen Umgebung nach. Dabei handelt es sich um ein vollständiges Betriebssystem, welches sich exakt so verhält, als sei es direkt auf der Hardware installiert.}
}
\newglossaryentry{glos:session}{name=Session,
	description={Eine Session ist eine stehende Verbindung eines Clients mit einem Server. Da das Internet auf dem zustandslosen Protokoll HTTP (Hypertext Transfer Protocol) basiert muss die Sitzung auf der Anwendungsschicht\footcite{OSI-Modell__Wikipedia_2015-07-30} etabliert werden. Dazu wird bei der ersten Kommunikation eine eindeutige Kennung vom Server an den Client übermittelt, welche dieser bei jedem Verbindungsaufbau zurück übermittelt.}
}
\newglossaryentry{glos:httpcookie}{name=HTTP-Cookie,
	description={Ein (HTTP-)Cookie ist eine Information, die der Server über den Browser auf dem Computer des Betrachters abspeichert. Die Cookie-Daten werden vom Browser bei jeder Anfrage an den Server zurück übermittelt.}
}
\newglossaryentry{glos:ide}{name=IDE,
	description={IDE steht für Integrated development environment und ist eine Sammlung von Anwendungsprogrammen, mit denen Software möglichst ohne Medienbrüche entwickelt werden kann.}
}
\newglossaryentry{glos:vcs}{name=Versionierungssystem,
	description={Wird zur Erfassung von Änderungen verwendet. Für jede Änderung wird eine Version angelegt wodurch diese auch rückgängig gemacht werden kann. Wenn mehrere Personen gleichzeitig an einem System arbeiten kann auch überprüft werden wer welche Anpassung vorgenommen hat.}
}
\newglossaryentry{glos:iis}{name=IIS,
	description={Die Internet Information Services (IIS) ist ein Web-, Mail-, FTP-, WebDAV-Server von Microsoft und wird benötigt um ASP- oder .NET Applikationen auszuführen.}
}

\newglossaryentry{glos:html}{name=HTML,
	description={Hypertext Markup Language (HTML) ist eine textbasierte Auszeichnungssprache welche für die erstellung von Webseiten verwendet wird. Die HTML-Dateien sind die Grundlagen des Internets und werden vom Browser dargestellt.}
}

%\newglossaryentry{glos:failureSafetyLabel}{name=Ausfallsicherheit,
%	description={Mit der Ausfallsicherheit wird die minimale zeitliche Erreichbarkeit (resp. maximale Ausfallzeit) eines Systems angegeben. Ist diese Ausfallzeit sehr gering spricht man von Hochverfügbarkeit (High Availability), dazu ist mindestens eine Verfügbarkeit von 99.9 \% nötig.
%	Die Verfügbarkeit berechnet man wie folgt:
%	\begin{gather*}
%		\text{Verfügbarkeit} = (1- \frac{\text{Ausfallzeit}}{\text{Periode}}) * 100\\
%		\text{Ausfallzeit} = (1 - \frac{\text{Verfügbarkeit}}{100}) * \text{Periode}
%	\end{gather*}}
%}
%\newglossaryentry{glos:thinClientLabel}{
%  name=Thin Client,
%  plural={Thin Clients},
%  description={Ein Thin Client ist ein günstiger, rechen-schwacher Computer. Er wird dazu verwendet, um Arbeiten zu erledigen die auf einem rechen-starken Server statt finden. Ein Thin Client übernimmt hauptsächlich die Bereitstellung von }
%}



% use it with \gls{glos:DVD}
% use plural with \glspl{thinClientLabel}
