\chapter{Einleitung}

\section{Ausgangslage}
Alle 3 bis 4 Wochen wird auf der Webseite von travelwindow AG eine neue Version eingespielt. Dies verlangt jedes Mal ein ausführliches Testen, was bislang immer manuell durchgeführt wurde. Dies ist eine sehr monotone und repetitive Arbeit, da die selben Testfälle auf unterschiedlichen Browsern überprüft werden müssen und dadurch oft Fehler übersehen werden. Der Testaufwand würde sich auf 3 bis 4 Tage belaufen, wenn alle kombinierbaren Möglichkeiten von Browser, Sprache und Produkten ausführlich geprüft werden müssen.

\section{Ziele der Arbeit}
\label{sec:desc:targets}
Die Einführung von automatisierten Integrationstests sollen die repetitiven Testarbeiten minimieren. Diese sollen mittels Selenium umgesetzt werden.
Das Ziel der Arbeit ist es, dass die Testpersonen bei jeder neuen Version nur noch testen müssen, was neu umgesetzt wurde.  Dadurch sollen mehr Programmfehler aufgedeckt und menschliche Fehler ausgeschlossen werden.
Später sollte durch die Automatisierung auch ein Continous Delivery ermöglicht werden.

\section{Aufgabenstellung}
In der Arbeit werden während vier Phasen die automatisierbaren Testfälle entwickelt. 

\subsection{Recherche}
Zu Beginn werden die einzusetzenden Frameworks, welche für die Entwicklung und Ausführung der Testfälle benötigt werden, evaluiert.

\subsection{Anforderungen und Analyse}
Als Grundlage für die Konzeptionierung werden bestmöglichst alle Testfälle spezifiziert. Diese sollen die Funktionalität der Webseite vollständig abdecken. Der Product Owner wird die Testfälle priorisieren, sodass entschieden werden kann nach welcher Reihenfolge die Testfälle konzeptioniert und implementiert werden können.

\subsection{Konzept}
Die Testfälle, welche während dieses Projektes umgesetzt werden, müssen aus spezifiziert werden. Es sind Vorbedingungen, Testabläufe, Soll-Resultate und Nachbedingungen zu definieren. Zusätzlich wird die gesamte Infrastruktur beschrieben.

\subsection{Umsetzung}
Die Testfälle werden implementiert und schlussendlich in den Application Lifecycle eingebunden, damit sie automatisiert und zu definierten Zeitpunkten durchgeführt werden können.
Je nach Anzahl der Testfälle werden alle oder eine Untermenge davon, in Abhängigkeit der erwähnten Priorisierung, umgesetzt.

\section{Erwartete Resultate}
Als Testvorbereitung werden die Testfälle spezifiziert und dokumentiert. Anschliessend werden verschiedene Testframeworks evaluiert und eine Architekturbeschreibung der Testinfrastruktur erstellt.
Es werden alle möglichen Testfälle spezifiziert. Diese werden sehr umfangreich ausfallen, weshalb Vorbedingungen, Testabläufe, Soll-Resultate und Nachbedingungen nur für jene Tests definiert werden, welche auch tatsächlich umgesetzt werden sollen. Die zu implementierenden Tests werden von der gesetzten Priorisierung bestimmt.

Der nächste Schritt ist die Implementierung der ausformulierten Tests. Dazu wird die Testinfrastruktur erstellt und die Tests programmiert. Ein Vergleich mit den Ist-Resultaten ist nicht nötig, da dies direkt von den automatischen Tests vorgenommen wird.

Schlussendlich müssen die Tests noch in den Application Lifcecycle integriert werden.

\begin{itemize}
\item Testfallspezitifkation inkl. Priorisierung
\item Evaluation von Testframeworks
\item Architekturbeschreibung der Testinfrastruktur
\item Aufsetzen der Testinfrastruktur
\item Implementierte Testfälle
\item Testfallintegration in den bestehenden Application Lifecycle
\end{itemize}