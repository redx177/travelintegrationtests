\chapter{Beschreibung der Aufgabe}

\section{Ausgangslage}
Alle 1-2 Monaten wird auf der Webseite von Travelwindow AG eine neue Version eingespielt. Dies verlangt jedes mal ein ausführliches Testen, was bislang immer manuell durchgeführt wurde. Da dies eine sehr monotone und repetitive Arbeit ist, werden auch oft Fehler übersehen und für verschiedene Browsers wird auch nicht getestet. 

\section{Ziele der Arbeit}
\label{sec:desc:targets}
Die Einführung von automatischen Integrationstests sollen die repetitiven Testarbeiten minimieren.

Ziel ist es, dass die Testpersonen bei jeder neuen Version nur noch testen müssen, was neu umgesetzt wurde.  Dadurch sollen mehr Programmfehler aufgedeckt und menschliche Fehler ausgeschlossen werden.

Später sollte durch die Automatisierung auch ein Continous Delivery ermöglicht werden.

\section{Aufgabenstellung}
In der Arbeit werden automatisierbare Testfälle entwickelt. 
Zuvor müssen diese spezifiziert und in Absprache mit Travelwindow AG priorisiert werden. Zusätzlich sind die einzusetzenden Frameworks und Architekturen zu evaluieren und dokummentieren.

Schlussendlich müssen die Tests in den Application Lifecycle eingebaut werden, so dass sie automatisiert, nach zu definierenden Bedingungen, durchgeführt werden.

Je nach Anzahl der Testfälle werden alle oder ein Untermenge davon, in Abhängigkeit der erwähnten Priorisierung, umgesetzt.

\section{Erwartete Resultate}
\begin{itemize}
\item Testfallspezitifkation inkl. Priorisierung
\item Evaluation von Testframeworks
\item Architekturbeschreibung der Testinfrastruktur
\item Aufsetzen der Testinfrastruktur
\item Implementierte Testfälle
\item Testfallintegration in den bestehenden Application Lifecycle
\item Tests sollen auf verschiedenen Browsern ausgeführt werden.
\end{itemize}