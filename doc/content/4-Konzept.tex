
\chapter{Konzept}
Dieses Kapitel spezifiziert die umzusetzenden Testfälle. Diese beinhalten die Ausführung, die erwarteten Resultate sowie allfällige Nachbedingungen. Die Vorbedingungen sind für alle Testfälle gleich, da jeder in einer neuen virtuellen Maschine (siehe \Gls{glos:virtualMachine}) gestartet wird und somit keine \Glspl{glos:session} und \Glspl{glos:httpcookie} vorhanden sind.

Um die Kosten zu minimieren werden möglichst wenig Testfälle spezifiziert, jedoch trotzdem die gesamte Funktionalität der Webseite überprüft werden kann.

Eine Übersicht der Testfälle wurde im Wiki von Hotelplan erstellt und wird zum Verständniss dort gepflegt. Die Liste ist im \cref{app:Testfälle} \nameref{app:Testfälle} unter "`Test Cases"' beigefügt.

\section{Übersicht}
Die Seite der \textit{travelwindow AG} besteht aus drei Bereichen (auch Engines genannt):
\begin{itemize}
\item Citytrip (Flug \& Hotel)
\item Flug
\item Hotel
\end{itemize}

Die Tests werden in drei Kategorien unterteilt:
\begin{itemize}
\item Happy Path
\item Smoke Tests
\item Main Tests
\end{itemize}

Wenn die Tests ausgeführt werden, sollen als erstes die Happy Paths\footcite{Happy_path_-_Wikipedia_the_free_encyclopedia_2015-07-30} überprüft werden. Dabei wird der gesamte Such- und Buchungungsprozess pro Engine einmal durchlaufen, ohne eine Auswahl zu treffen oder auf Fehler zu überprüfen. Der Test ist erfolgreich wenn die letzte Seite des Buchungsprozesses erreicht werden kann.

Danach folgen die Smoke- und Main Tests. Die Smoke Tests führen eine Suche für die in der \cref{sec:Recherche:Zielgruppe:top10} \nameref{sec:Recherche:Zielgruppe:top10} definierten top 10 Detinationen aus und überprüfen, ob Resultate geliefert werden.

Die Main Tests überprüfen schlussendlich die Funktionalität der Webseite.

\subsection{Kunde \& Passagierangaben}
\label{sec:Konzept:Übersicht:Angaben}
Im Buchungsprozess müssen die Kunden- \& Passagierangaben ausgefüllt werden. Für alle Tests werden die folgenden Daten verwendet.

Kundendaten:
\begin{itemize}
\item Vorname: Han
\item Name: Sola
\item Strasse \& Nummer: Bederstrasse 66
\item Postleitzahl: 8002
\item Ort: Zürich
\item E-Mail Adresse: info@travel.ch
\item Telefonnummer: +41 44 200 26 26 
\end{itemize}

Passagierangaben (Geschlecht, Vorname, Name, Geburtsdatum):
\begin{itemize}
\item Passagier 1: Männlich, Han, Solo, 24.03.1980
\item Passagier 2: Weiblich, Leia Organa, Solo, 12.02.1978
\item Passagier 3: Männlich, Luke, Skywalker, 07.08.1985
\end{itemize}

\section{Happy Path}
\subsection{Citytrip}
\begin{table}[H] 
	\caption{Testspezifikation: Happy Path - Citytrip}
	\centering
	\rowcolors{1}{tablebodycolor}{tablerowcolor}
		
	\begin{tabularx}{0.9\textwidth}{ | l | X | } 
		\hline 
		\textbf{Seite} & \textbf{Aktion} \\ \hline 
		\multirow{1}{*}{-} & Gehe auf die Startseite \\ \hline
		\multirow{1}{*}{Startseite} & Destinationsfeld: Leeren \\ \cline{2-2}
		& Destinationsfeld: Gebe "`ber"' ein \\ \cline{2-2}
		& Destinationsfeld: Warte bis die Vorschläge erscheinen \\ \cline{2-2}
		& Destinationsfeld: Klicke auf den ersten Vorschlag \\ \cline{2-2}
		& Zimmerangabe: Klicke auf das Feld \\ \cline{2-2}
		& Zimmerangabe: Wähle "`1 Erwachsenen"' aus \\ \cline{2-2}
		& Zimmerangabe: Wähle "`1 Kind"' aus \\ \cline{2-2}
		& Zimmerangabe: Wähle für das Geburtsdatum jeweils das erste Element aus den Dropdowns aus.  \\ \cline{2-2}
		& Zimmerangabe: Füge ein Zimmer hinzu \\ \cline{2-2}
		& Zimmerangabe: Wähle "`1 Erwachsenen"' aus \\ \cline{2-2}
		& Datumsangabe: Klicke auf das Abreisedatum \\ \cline{2-2}
		& Datumsangabe: Gehe 3 Monate in die Zukunft \\ \cline{2-2}
		& Datumsangabe: Wähle das erste Datum an \\ \cline{2-2}
		& Datumsangabe: Klicke auf das Rückreisedatum \\ \cline{2-2}
		& Datumsangabe: Wähle das dritte Datum an \\ \cline{2-2}
		& Klicke auf den Suchen Button \\ \hline
		
		\multirow{1}{*}{Hotel Suchresultatseite} & Warte bis die Suchresultate dargestellt werden \\ \cline{2-2}
		& Wähle das erste Hotel aus \\ \hline
		
		\multirow{1}{*}{Hotel Konfigurationsseite} & Warte bis der "`Weiter"' Button anwählbar ist \\ \cline{2-2}
		& Klicke auf den "`Weiter"' Button \\ \hline
				
		\multirow{1}{*}{Flug Konfigurationsseite} & Warte bis die Seite geladen ist \\ \cline{2-2}
		& Warte bis der "`Weiter"' Button anwählbar ist \\ \cline{2-2}
		& Klicke auf den "`Weiter"' Button \\ \hline
		
		\multirow{1}{*}{Checkout: Passagierangabe} & Warte bis der Warenkorb geladen ist \\ \cline{2-2}
		& Fülle die Passagierangaben wie im \cref{sec:Konzept:Übersicht:Angaben} \nameref{sec:Konzept:Übersicht:Angaben} beschrieben ab. \\ \cline{2-2}
		& Fülle die Kundendaten wie im \cref{sec:Konzept:Übersicht:Angaben} \nameref{sec:Konzept:Übersicht:Angaben} beschrieben ab. \\ \cline{2-2}
		& Klicke auf den "`Weiter"' Button \\ \hline
		
		\multirow{1}{*}{Checkout: Bezahlart} & Warte bis die Seite geladen ist \\ \cline{2-2}
		& Warte bis der Warenkorb geladen ist \\ \cline{2-2}
		& Klicke auf den "`Weiter"' Button \\ \hline
		
		\multirow{1}{*}{Checkout: Übersicht} & Warte bis die Seite geladen ist \\ \cline{2-2}
		& Warte bis der Warenkorb geladen ist \\ \hline
	\end{tabularx} 
\end{table}

\subsection{Flight}
\begin{table}[H] 
	\caption{Testspezifikation: Happy Path - Flight}
	\centering
	\rowcolors{1}{tablebodycolor}{tablerowcolor}
		
	\begin{tabularx}{0.9\textwidth}{ | l | X | } 
		\hline 
		\textbf{Seite} & \textbf{Aktion} \\ \hline 
		\multirow{1}{*}{-} & Gehe auf die Startseite \\ \hline
		\multirow{1}{*}{Startseite} & Abflugsort Destinationsfeld: Leeren \\ \cline{2-2}
		& Abflugsort Destinationsfeld: Gebe "`zurich"' ein \\ \cline{2-2}
		& Abflugsort Destinationsfeld: Warte bis die Vorschläge erscheinen \\ \cline{2-2}
		& Abflugsort Destinationsfeld: Klicke auf den ersten Vorschlag \\ \cline{2-2}
		& Ankunftsort Destinationsfeld: Leeren \\ \cline{2-2}
		& Ankunftsort Destinationsfeld: Gebe "`ber"' ein \\ \cline{2-2}
		& Ankunftsort Destinationsfeld: Warte bis die Vorschläge erscheinen \\ \cline{2-2}
		& Ankunftsort Destinationsfeld: Klicke auf den ersten Vorschlag \\ \cline{2-2}
		& Zimmerangabe: Klicke auf das Feld \\ \cline{2-2}
		& Zimmerangabe: Wähle "`2 Erwachsenen"' aus \\ \cline{2-2}
		& Zimmerangabe: Wähle "`1 Kind"' aus \\ \cline{2-2}
		& Zimmerangabe: Wähle für das Geburtsdatum jeweils das erste Element aus den Dropdowns aus.  \\ \cline{2-2}
		& Datumsangabe: Klicke auf das Abreisedatum \\ \cline{2-2}
		& Datumsangabe: Gehe 3 Monate in die Zukunft \\ \cline{2-2}
		& Datumsangabe: Wähle das erste Datum an \\ \cline{2-2}
		& Datumsangabe: Klicke auf das Rückreisedatum \\ \cline{2-2}
		& Datumsangabe: Wähle das dritte Datum an \\ \cline{2-2}
		& Klicke auf den Suchen Button \\ \hline
				
		\multirow{1}{*}{Flug Suchresultatseite} & Warte bis die Suchresultate dargestellt werden \\ \cline{2-2}
		& Wähle den ersten Flug aus \\ \hline
		
		\multirow{1}{*}{Checkout: Passagierangabe} & Warte bis der Warenkorb geladen ist \\ \cline{2-2}
		& Fülle die Passagierangaben wie im \cref{sec:Konzept:Übersicht:Angaben} \nameref{sec:Konzept:Übersicht:Angaben} beschrieben ab. \\ \cline{2-2}
		& Fülle die Kundendaten wie im \cref{sec:Konzept:Übersicht:Angaben} \nameref{sec:Konzept:Übersicht:Angaben} beschrieben ab. \\ \cline{2-2}
		& Klicke auf den "`Weiter"' Button \\ \hline
		
		\multirow{1}{*}{Checkout: Bezahlart} & Warte bis die Seite geladen ist \\ \cline{2-2}
		& Warte bis der Warenkorb geladen ist \\ \cline{2-2}
		& Klicke auf den "`Weiter"' Button \\ \hline
		
		\multirow{1}{*}{Checkout: Übersicht} & Warte bis die Seite geladen ist \\ \cline{2-2}
		& Warte bis der Warenkorb geladen ist \\ \hline
	\end{tabularx} 
\end{table}

\subsection{Hotel}
\begin{table}[H] 
	\caption{Testspezifikation: Happy Path - Hotel}
	\centering
	\rowcolors{1}{tablebodycolor}{tablerowcolor}
		
	\begin{tabularx}{0.9\textwidth}{ | l | X | } 
		\hline 
		\textbf{Seite} & \textbf{Aktion} \\ \hline 
		\multirow{1}{*}{-} & Gehe auf die Startseite \\ \hline
		\multirow{1}{*}{Startseite} & Destinationsfeld: Leeren \\ \cline{2-2}
		& Destinationsfeld: Gebe "`ber"' ein \\ \cline{2-2}
		& Destinationsfeld: Warte bis die Vorschläge erscheinen \\ \cline{2-2}
		& Destinationsfeld: Klicke auf den ersten Vorschlag \\ \cline{2-2}
		& Zimmerangabe: Klicke auf das Feld \\ \cline{2-2}
		& Zimmerangabe: Wähle "`1 Erwachsenen"' aus \\ \cline{2-2}
		& Zimmerangabe: Wähle "`1 Kind"' aus \\ \cline{2-2}
		& Zimmerangabe: Wähle für das Geburtsdatum jeweils das erste Element aus den Dropdowns aus.  \\ \cline{2-2}
		& Zimmerangabe: Füge ein Zimmer hinzu \\ \cline{2-2}
		& Zimmerangabe: Wähle "`1 Erwachsenen"' aus \\ \cline{2-2}
		& Datumsangabe: Klicke auf das Abreisedatum \\ \cline{2-2}
		& Datumsangabe: Gehe 3 Monate in die Zukunft \\ \cline{2-2}
		& Datumsangabe: Wähle das erste Datum an \\ \cline{2-2}
		& Datumsangabe: Klicke auf das Rückreisedatum \\ \cline{2-2}
		& Datumsangabe: Wähle das dritte Datum an \\ \cline{2-2}
		& Klicke auf den Suchen Button \\ \hline
		
		\multirow{1}{*}{Hotel Suchresultatseite} & Warte bis die Suchresultate dargestellt werden \\ \cline{2-2}
		& Wähle das erste Hotel aus \\ \hline
		
		\multirow{1}{*}{Hotel Konfigurationsseite} & Warte bis der "`Weiter"' Button anwählbar ist \\ \cline{2-2}
		& Klicke auf den "`Weiter"' Button \\ \hline
		
		\multirow{1}{*}{Checkout: Passagierangabe} & Warte bis der Warenkorb geladen ist \\ \cline{2-2}
		& Fülle die Passagierangaben wie im \cref{sec:Konzept:Übersicht:Angaben} \nameref{sec:Konzept:Übersicht:Angaben} beschrieben ab. \\ \cline{2-2}
		& Fülle die Kundendaten wie im \cref{sec:Konzept:Übersicht:Angaben} \nameref{sec:Konzept:Übersicht:Angaben} beschrieben ab. \\ \cline{2-2}
		& Klicke auf den "`Weiter"' Button \\ \hline
		
		\multirow{1}{*}{Checkout: Bezahlart} & Warte bis die Seite geladen ist \\ \cline{2-2}
		& Warte bis der Warenkorb geladen ist \\ \cline{2-2}
		& Klicke auf den "`Weiter"' Button \\ \hline
		
		\multirow{1}{*}{Checkout: Übersicht} & Warte bis die Seite geladen ist \\ \cline{2-2}
		& Warte bis der Warenkorb geladen ist \\ \hline
	\end{tabularx} 
\end{table}

\section{Smoke Tests}
Smoke Tests führen eine Aktion durch und schauen lediglich, ob kein Fehler aufgetreten ist. Der Begriff rührt daher, dass eine Maschine gestartet und getestet wurde, ob sie nicht beginnt zu rauchen\footcite{Smoke_testing_software_-_Wikipedia_the_free_encyclopedia_2015-08-01}

Bei den folgenden Smoke Tests werden pro Engine die jeweiligen Top 10 (siehe \cref{sec:Recherche:Zielgruppe:top10} \nameref{sec:Recherche:Zielgruppe:top10}) Destinationen gesucht und überprüft, ob Suchresultate geliefert werden.

Die Tests sind pro Engine immer gleich aufgebaut. Sprich die Suchparameter sind dieselben und auch die Verifikation ob kein Fehler aufgetreten ist äquivalent. Das einzige was sich unterscheidet sind die Destinationen. Daher wird jeweils nur der Test für die erste Destination ausführlich spezifiziert. Für die restlichen 9 gilt der selbe Ablauf.

\subsection{Citytrip}
\begin{table}[H] 
	\caption{Testspezifikation: Smoke Test - Citytrip}
	\centering
	\rowcolors{1}{tablebodycolor}{tablerowcolor}
		
	\begin{tabularx}{0.9\textwidth}{ | l | X | } 
		\hline 
		\textbf{Seite} & \textbf{Aktion} \\ \hline 
		\multirow{1}{*}{-} & Gehe auf die Startseite \\ \hline
		\multirow{1}{*}{Startseite} & Destinationsfeld: Leeren \\ \cline{2-2}
		& Destinationsfeld: Gebe "`Wien"' ein \\ \cline{2-2}
		& Destinationsfeld: Warte bis die Vorschläge erscheinen \\ \cline{2-2}
		& Destinationsfeld: Klicke auf den ersten Vorschlag \\ \cline{2-2}
		& Zimmerangabe: Klicke auf das Feld \\ \cline{2-2}
		& Zimmerangabe: Wähle "`2 Erwachsenen"' aus \\ \cline{2-2}
		& Datumsangabe: Klicke auf das Abreisedatum \\ \cline{2-2}
		& Datumsangabe: Gehe 3 Monate in die Zukunft \\ \cline{2-2}
		& Datumsangabe: Wähle das erste Datum an \\ \cline{2-2}
		& Datumsangabe: Klicke auf das Rückreisedatum \\ \cline{2-2}
		& Datumsangabe: Wähle das dritte Datum an \\ \cline{2-2}
		& Klicke auf den Suchen Button \\ \hline
		
		\multirow{1}{*}{Hotel Suchresultatseite} & Warte bis die Suchresultate dargestellt werden \\ \cline{2-2}
		& \textbf{Verifikation:} Es muss mind. ein Hotel vorhanden sein. \\ \hline
	\end{tabularx} 
\end{table}

\subsection{Flight}
\begin{table}[H] 
	\caption{Testspezifikation: Smoke Test - Flight}
	\centering
	\rowcolors{1}{tablebodycolor}{tablerowcolor}
		
	\begin{tabularx}{0.9\textwidth}{ | l | X | } 
		\hline 
		\textbf{Seite} & \textbf{Aktion} \\ \hline 
		\multirow{1}{*}{-} & Gehe auf die Startseite \\ \hline
		\multirow{1}{*}{Startseite} & Abflugsort Destinationsfeld: Leeren \\ \cline{2-2}
		& Abflugsort Destinationsfeld: Gebe "`Zurich"' ein \\ \cline{2-2}
		& Abflugsort Destinationsfeld: Warte bis die Vorschläge erscheinen \\ \cline{2-2}
		& Abflugsort Destinationsfeld: Klicke auf den ersten Vorschlag \\ \cline{2-2}
		& Ankunftsort Destinationsfeld: Leeren \\ \cline{2-2}
		& Ankunftsort Destinationsfeld: Gebe "`Palma de Mallorca"' ein \\ \cline{2-2}
		& Ankunftsort Destinationsfeld: Warte bis die Vorschläge erscheinen \\ \cline{2-2}
		& Ankunftsort Destinationsfeld: Klicke auf den ersten Vorschlag \\ \cline{2-2}
		& Zimmerangabe: Klicke auf das Feld \\ \cline{2-2}
		& Zimmerangabe: Wähle "`2 Erwachsenen"' aus \\ \cline{2-2}
		& Datumsangabe: Klicke auf das Abreisedatum \\ \cline{2-2}
		& Datumsangabe: Gehe 3 Monate in die Zukunft \\ \cline{2-2}
		& Datumsangabe: Wähle das erste Datum an \\ \cline{2-2}
		& Datumsangabe: Klicke auf das Rückreisedatum \\ \cline{2-2}
		& Datumsangabe: Wähle das dritte Datum an \\ \cline{2-2}
		& Klicke auf den Suchen Button \\ \hline
				
		\multirow{1}{*}{Flug Suchresultatseite} & Warte bis die Suchresultate dargestellt werden \\ \cline{2-2}
		& \textbf{Verifikation:} Es muss mind. ein Flug vorhanden sein. \\ \hline
	\end{tabularx} 
\end{table}

\subsection{Hotel}
\begin{table}[H] 
	\caption{Testspezifikation: Smoke Test - Hotel}
	\centering
	\rowcolors{1}{tablebodycolor}{tablerowcolor}
		
	\begin{tabularx}{0.9\textwidth}{ | l | X | } 
			\hline 
			\textbf{Seite} & \textbf{Aktion} \\ \hline 
			\multirow{1}{*}{-} & Gehe auf die Startseite \\ \hline
			\multirow{1}{*}{Startseite} & Destinationsfeld: Leeren \\ \cline{2-2}
			& Destinationsfeld: Gebe "`ber"' ein \\ \cline{2-2}
			& Destinationsfeld: Warte bis die Vorschläge erscheinen \\ \cline{2-2}
			& Destinationsfeld: Klicke auf den ersten Vorschlag \\ \cline{2-2}
			& Zimmerangabe: Klicke auf das Feld \\ \cline{2-2}
			& Zimmerangabe: Wähle "`1 Erwachsenen"' aus \\ \cline{2-2}
			& Zimmerangabe: Wähle "`1 Kind"' aus \\ \cline{2-2}
			& Zimmerangabe: Wähle für das Geburtsdatum jeweils das erste Element aus den Dropdowns aus.  \\ \cline{2-2}
			& Zimmerangabe: Füge ein Zimmer hinzu \\ \cline{2-2}
			& Zimmerangabe: Wähle "`1 Erwachsenen"' aus \\ \cline{2-2}
			& Datumsangabe: Klicke auf das Abreisedatum \\ \cline{2-2}
			& Datumsangabe: Gehe 3 Monate in die Zukunft \\ \cline{2-2}
			& Datumsangabe: Wähle das erste Datum an \\ \cline{2-2}
			& Datumsangabe: Klicke auf das Rückreisedatum \\ \cline{2-2}
			& Datumsangabe: Wähle das dritte Datum an \\ \cline{2-2}
			& Klicke auf den Suchen Button \\ \hline
					
			\multirow{1}{*}{Hotel Suchresultatseite} & Warte bis die Suchresultate dargestellt werden \\ \cline{2-2}
			& \textbf{Verifikation:} Es muss mind. ein Hotel vorhanden sein. \\ \hline
	\end{tabularx} 
\end{table}

\section{Main Tests}