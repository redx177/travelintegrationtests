
\chapter{Konzept}
\label{sec:konzept}
Dieses Kapitel spezifiziert die umzusetzenden Testfälle. Diese beinhalten die Ausführung, die erwarteten Resultate sowie allfällige Nachbedingungen. Die Vorbedingungen sind für alle Testfälle gleich, da jeder in einer neuen virtuellen Maschine (siehe \Gls{glos:virtualMachine}) gestartet wird und somit keine \Glspl{glos:session} und \Glspl{glos:httpcookie} vorhanden sind.

Um die Kosten zu minimieren werden möglichst wenig Testfälle spezifiziert, jedoch trotzdem die gesamte Funktionalität der Webseite überprüft werden kann. Weitere Details dazu sind im \cref{sec:analyse} \nameref{sec:analyse} aufgeführt.

Eine Übersicht der Testfälle wurde im Wiki von Hotelplan erstellt und wird zum Verständnis dort gepflegt. Die Liste ist im \cref{app:Testfälle} \nameref{app:Testfälle} unter "`Test Cases"' beigefügt.

\section{Übersicht}
Die Seite der \textit{travelwindow AG} besteht aus drei Bereichen (auch Engines genannt):
\begin{itemize}
\item Citytrip (Flug \& Hotel)
\item Flug
\item Hotel
\end{itemize}

Die Tests werden in drei Kategorien unterteilt:
\begin{itemize}
\item Happy Path
\item Smoke Tests
\item Main Tests
\end{itemize}

Wenn die Tests ausgeführt werden, sollen als erstes die Happy Paths\footcite{Happy_path_-_Wikipedia_the_free_encyclopedia_2015-07-30} überprüft werden. Dabei wird der gesamte Such- und Buchungungsprozess pro Engine einmal durchlaufen, ohne eine Auswahl zu treffen oder auf Fehler zu überprüfen. Der Test ist erfolgreich wenn die letzte Seite des Buchungsprozesses erreicht werden kann. Damit wird sichergestellt dass die Webseite im standartdurchlauf keine Fehler produziert.

Danach folgen die Smoke- und Main Tests. Die Smoke Tests führen eine Suche für die in der \cref{sec:Recherche:Zielgruppe:top10} \nameref{sec:Recherche:Zielgruppe:top10} definierten Destinationen aus und überprüfen, ob Resultate geliefert werden.

Die Main Tests überprüfen schlussendlich die Funktionalität (siehe \cref{app:Funktionalitäten} \nameref{app:Funktionalitäten}) der Webseite.

\subsection{Kunde \& Passagierangaben}
\label{sec:Konzept:Übersicht:Angaben}
Im Buchungsprozess müssen die Kunden- \& Passagierangaben ausgefüllt werden. Für alle Tests werden die folgenden Daten verwendet.
 \\
 \\
Kundendaten:
\begin{itemize}
\item Vorname: Han
\item Name: Sola
\item Strasse \& Nummer: Bederstrasse 66
\item Postleitzahl: 8002
\item Ort: Zürich
\item E-Mail Adresse: info@travel.ch
\item Telefonnummer: +41 44 200 26 26 
\end{itemize}

Passagierangaben (Geschlecht, Vorname, Name, Geburtsdatum):
\begin{itemize}
\item Passagier 1: Männlich, Han, Solo, 24.03.1980
\item Passagier 2: Weiblich, Leia Organa, Solo, 12.02.1978
\item Passagier 3: Männlich, Luke, Skywalker, 07.08.1985
\end{itemize}

\section{Happy Path}
Der Happy Path testet, ob beim standard Szenario keine Fehler auftreten. Es werden demnach jede Engine einmal durchlaufen mit gültigen Angaben. Dies soll überprüfen ob keine grundlegenden Fehler vorhanden sind.
\subsection{Citytrip}
\begin{table}[H] 
	\caption{Testspezifikation: Happy Path - Citytrip}
	\centering
	\rowcolors{1}{tablebodycolor}{tablerowcolor}
		
	\begin{tabularx}{0.9\textwidth}{ | l | X | } 
		\hline 
		\textbf{Seite} & \textbf{Aktion} \\ \hline 
		\multirow{1}{*}{-} & Gehe auf die Startseite \\ \hline
		\multirow{1}{*}{Startseite} & Destinationsfeld: Leeren \\ \cline{2-2}
		& Destinationsfeld: Gebe "`ber"' ein \\ \cline{2-2}
		& Destinationsfeld: Warte bis die Vorschläge erscheinen \\ \cline{2-2}
		& Destinationsfeld: Klicke auf den ersten Vorschlag \\ \cline{2-2}
		& Zimmerangabe: Klicke auf das Feld \\ \cline{2-2}
		& Zimmerangabe: Wähle "`1 Erwachsenen"' aus \\ \cline{2-2}
		& Zimmerangabe: Wähle "`1 Kind"' aus \\ \cline{2-2}
		& Zimmerangabe: Wähle für das Geburtsdatum jeweils das erste Element aus den Dropdowns aus.  \\ \cline{2-2}
		& Zimmerangabe: Füge ein Zimmer hinzu \\ \cline{2-2}
		& Zimmerangabe: Wähle "`1 Erwachsenen"' aus \\ \cline{2-2}
		& Datumsangabe: Klicke auf das Abreisedatum \\ \cline{2-2}
		& Datumsangabe: Gehe 3 Monate in die Zukunft \\ \cline{2-2}
		& Datumsangabe: Wähle das erste Datum an \\ \cline{2-2}
		& Datumsangabe: Klicke auf das Rückreisedatum \\ \cline{2-2}
		& Datumsangabe: Wähle das dritte Datum an \\ \cline{2-2}
		& Klicke auf den Suchen Button \\ \hline
		
		\multirow{1}{*}{Hotel Suchresultatseite} & Warte bis die Suchresultate dargestellt werden \\ \cline{2-2}
		& Wähle das erste Hotel aus \\ \hline
		
		\multirow{1}{*}{Hotel Konfigurationsseite} & Warte bis der "`Weiter"' Button anwählbar ist \\ \cline{2-2}
		& Klicke auf den "`Weiter"' Button \\ \hline
				
		\multirow{1}{*}{Flug Konfigurationsseite} & Warte bis die Seite geladen ist \\ \cline{2-2}
		& Warte bis der "`Weiter"' Button anwählbar ist \\ \cline{2-2}
		& Klicke auf den "`Weiter"' Button \\ \hline
		
		\multirow{1}{*}{Checkout: Passagierangabe} & Warte bis der Warenkorb geladen ist \\ \cline{2-2}
		& Fülle die Passagierangaben wie im \cref{sec:Konzept:Übersicht:Angaben} \nameref{sec:Konzept:Übersicht:Angaben} beschrieben ab. \\ \cline{2-2}
		& Fülle die Kundendaten wie im \cref{sec:Konzept:Übersicht:Angaben} \nameref{sec:Konzept:Übersicht:Angaben} beschrieben ab. \\ \cline{2-2}
		& Klicke auf den "`Weiter"' Button \\ \hline
		
		\multirow{1}{*}{Checkout: Bezahlart} & Warte bis die Seite geladen ist \\ \cline{2-2}
		& Warte bis der Warenkorb geladen ist \\ \cline{2-2}
		& Klicke auf den "`Weiter"' Button \\ \hline
		
		\multirow{1}{*}{Checkout: Übersicht} & Warte bis die Seite geladen ist \\ \cline{2-2}
		& Warte bis der Warenkorb geladen ist \\ \hline
	\end{tabularx} 
\end{table}

\subsection{Flight}
\begin{table}[H] 
	\caption{Testspezifikation: Happy Path - Flight}
	\centering
	\rowcolors{1}{tablebodycolor}{tablerowcolor}
		
	\begin{tabularx}{0.9\textwidth}{ | l | X | } 
		\hline 
		\textbf{Seite} & \textbf{Aktion} \\ \hline 
		\multirow{1}{*}{-} & Gehe auf die Startseite \\ \hline
		\multirow{1}{*}{Startseite} & Abflugsort Destinationsfeld: Leeren \\ \cline{2-2}
		& Abflugsort Destinationsfeld: Gebe "`zurich"' ein \\ \cline{2-2}
		& Abflugsort Destinationsfeld: Warte bis die Vorschläge erscheinen \\ \cline{2-2}
		& Abflugsort Destinationsfeld: Klicke auf den ersten Vorschlag \\ \cline{2-2}
		& Ankunftsort Destinationsfeld: Leeren \\ \cline{2-2}
		& Ankunftsort Destinationsfeld: Gebe "`ber"' ein \\ \cline{2-2}
		& Ankunftsort Destinationsfeld: Warte bis die Vorschläge erscheinen \\ \cline{2-2}
		& Ankunftsort Destinationsfeld: Klicke auf den ersten Vorschlag \\ \cline{2-2}
		& Passagierangabe: Klicke auf das Feld \\ \cline{2-2}
		& Passagierangabe: Wähle "`2 Erwachsenen"' aus \\ \cline{2-2}
		& Passagierangabe: Wähle "`1 Kind"' aus \\ \cline{2-2}
		& Passagierangabe: Wähle für das Geburtsdatum jeweils das erste Element aus den Dropdowns aus.  \\ \cline{2-2}
		& Datumsangabe: Klicke auf das Abreisedatum \\ \cline{2-2}
		& Datumsangabe: Gehe 3 Monate in die Zukunft \\ \cline{2-2}
		& Datumsangabe: Wähle das erste Datum an \\ \cline{2-2}
		& Datumsangabe: Klicke auf das Rückreisedatum \\ \cline{2-2}
		& Datumsangabe: Wähle das dritte Datum an \\ \cline{2-2}
		& Klicke auf den Suchen Button \\ \hline
				
		\multirow{1}{*}{Flug Suchresultatseite} & Warte bis die Suchresultate dargestellt werden \\ \cline{2-2}
		& Wähle den ersten Flug aus \\ \hline
		
		\multirow{1}{*}{Checkout: Passagierangabe} & Warte bis der Warenkorb geladen ist \\ \cline{2-2}
		& Fülle die Passagierangaben wie im \cref{sec:Konzept:Übersicht:Angaben} \nameref{sec:Konzept:Übersicht:Angaben} beschrieben ab. \\ \cline{2-2}
		& Fülle die Kundendaten wie im \cref{sec:Konzept:Übersicht:Angaben} \nameref{sec:Konzept:Übersicht:Angaben} beschrieben ab. \\ \cline{2-2}
		& Klicke auf den "`Weiter"' Button \\ \hline
		
		\multirow{1}{*}{Checkout: Bezahlart} & Warte bis die Seite geladen ist \\ \cline{2-2}
		& Warte bis der Warenkorb geladen ist \\ \cline{2-2}
		& Klicke auf den "`Weiter"' Button \\ \hline
		
		\multirow{1}{*}{Checkout: Übersicht} & Warte bis die Seite geladen ist \\ \cline{2-2}
		& Warte bis der Warenkorb geladen ist \\ \hline
	\end{tabularx} 
\end{table}

\subsection{Hotel}
\begin{table}[H] 
	\caption{Testspezifikation: Happy Path - Hotel}
	\centering
	\rowcolors{1}{tablebodycolor}{tablerowcolor}
		
	\begin{tabularx}{0.9\textwidth}{ | l | X | } 
		\hline 
		\textbf{Seite} & \textbf{Aktion} \\ \hline 
		\multirow{1}{*}{-} & Gehe auf die Startseite \\ \hline
		\multirow{1}{*}{Startseite} & Destinationsfeld: Leeren \\ \cline{2-2}
		& Destinationsfeld: Gebe "`ber"' ein \\ \cline{2-2}
		& Destinationsfeld: Warte bis die Vorschläge erscheinen \\ \cline{2-2}
		& Destinationsfeld: Klicke auf den ersten Vorschlag \\ \cline{2-2}
		& Zimmerangabe: Klicke auf das Feld \\ \cline{2-2}
		& Zimmerangabe: Wähle "`1 Erwachsenen"' aus \\ \cline{2-2}
		& Zimmerangabe: Wähle "`1 Kind"' aus \\ \cline{2-2}
		& Zimmerangabe: Wähle für das Geburtsdatum jeweils das erste Element aus den Dropdowns aus.  \\ \cline{2-2}
		& Zimmerangabe: Füge ein Zimmer hinzu \\ \cline{2-2}
		& Zimmerangabe: Wähle "`1 Erwachsenen"' aus \\ \cline{2-2}
		& Datumsangabe: Klicke auf das Abreisedatum \\ \cline{2-2}
		& Datumsangabe: Gehe 3 Monate in die Zukunft \\ \cline{2-2}
		& Datumsangabe: Wähle das erste Datum an \\ \cline{2-2}
		& Datumsangabe: Klicke auf das Rückreisedatum \\ \cline{2-2}
		& Datumsangabe: Wähle das dritte Datum an \\ \cline{2-2}
		& Klicke auf den Suchen Button \\ \hline
		
		\multirow{1}{*}{Hotel Suchresultatseite} & Warte bis die Suchresultate dargestellt werden \\ \cline{2-2}
		& Wähle das erste Hotel aus \\ \hline
		
		\multirow{1}{*}{Hotel Konfigurationsseite} & Warte bis der "`Weiter"' Button anwählbar ist \\ \cline{2-2}
		& Klicke auf den "`Weiter"' Button \\ \hline
		
		\multirow{1}{*}{Checkout: Passagierangabe} & Warte bis der Warenkorb geladen ist \\ \cline{2-2}
		& Fülle die Passagierangaben wie im \cref{sec:Konzept:Übersicht:Angaben} \nameref{sec:Konzept:Übersicht:Angaben} beschrieben ab. \\ \cline{2-2}
		& Fülle die Kundendaten wie im \cref{sec:Konzept:Übersicht:Angaben} \nameref{sec:Konzept:Übersicht:Angaben} beschrieben ab. \\ \cline{2-2}
		& Klicke auf den "`Weiter"' Button \\ \hline
		
		\multirow{1}{*}{Checkout: Bezahlart} & Warte bis die Seite geladen ist \\ \cline{2-2}
		& Warte bis der Warenkorb geladen ist \\ \cline{2-2}
		& Klicke auf den "`Weiter"' Button \\ \hline
		
		\multirow{1}{*}{Checkout: Übersicht} & Warte bis die Seite geladen ist \\ \cline{2-2}
		& Warte bis der Warenkorb geladen ist \\ \hline
	\end{tabularx} 
\end{table}

\section{Smoke Tests}
\label{sec:konzept:smoketests}
Smoke Tests führen eine Aktion durch und schauen lediglich, ob kein Fehler aufgetreten ist. Der Begriff rührt daher, dass eine Maschine gestartet und getestet wurde, ob sie nicht beginnt zu rauchen\footcite{Smoke_testing_software_-_Wikipedia_the_free_encyclopedia_2015-08-01}

Bei den folgenden Smoke Tests werden pro Engine die jeweiligen Top 10 (siehe \cref{sec:Recherche:Zielgruppe:top10} \nameref{sec:Recherche:Zielgruppe:top10}) Destinationen gesucht und überprüft, ob Suchresultate geliefert werden.

Die Tests sind pro Engine immer gleich aufgebaut. Sprich die Suchparameter sind dieselben und auch die Verifikation ob kein Fehler aufgetreten ist äquivalent. Das einzige was sich unterscheidet sind die Destinationen. Daher wird jeweils nur der Test für die erste Destination ausführlich spezifiziert. Für die restlichen 9 gilt der selbe Ablauf.

\subsection{Citytrip}
\begin{table}[H] 
	\caption{Testspezifikation: Smoke Test - Citytrip}
	\centering
	\rowcolors{1}{tablebodycolor}{tablerowcolor}
		
	\begin{tabularx}{0.9\textwidth}{ | l | X | } 
		\hline 
		\textbf{Seite} & \textbf{Aktion} \\ \hline 
		\multirow{1}{*}{-} & Gehe auf die Startseite \\ \hline
		\multirow{1}{*}{Startseite} & Destinationsfeld: Leeren \\ \cline{2-2}
		& Destinationsfeld: Gebe "`Wien"' ein \\ \cline{2-2}
		& Destinationsfeld: Warte bis die Vorschläge erscheinen \\ \cline{2-2}
		& Destinationsfeld: Klicke auf den ersten Vorschlag \\ \cline{2-2}
		& Zimmerangabe: Klicke auf das Feld \\ \cline{2-2}
		& Zimmerangabe: Wähle "`2 Erwachsenen"' aus \\ \cline{2-2}
		& Datumsangabe: Klicke auf das Abreisedatum \\ \cline{2-2}
		& Datumsangabe: Gehe 3 Monate in die Zukunft \\ \cline{2-2}
		& Datumsangabe: Wähle das erste Datum an \\ \cline{2-2}
		& Datumsangabe: Klicke auf das Rückreisedatum \\ \cline{2-2}
		& Datumsangabe: Wähle das dritte Datum an \\ \cline{2-2}
		& Klicke auf den Suchen Button \\ \hline
		
		\multirow{1}{*}{Hotel Suchresultatseite} & Warte bis die Suchresultate dargestellt werden \\ \cline{2-2}
		& \textbf{Verifikation:} Es muss mind. ein Hotel vorhanden sein. \\ \hline
	\end{tabularx} 
\end{table}

\subsection{Flight}
\begin{table}[H] 
	\caption{Testspezifikation: Smoke Test - Flight}
	\centering
	\rowcolors{1}{tablebodycolor}{tablerowcolor}
		
	\begin{tabularx}{0.9\textwidth}{ | l | X | } 
		\hline 
		\textbf{Seite} & \textbf{Aktion} \\ \hline 
		\multirow{1}{*}{-} & Gehe auf die Startseite \\ \hline
		\multirow{1}{*}{Startseite} & Abflugsort Destinationsfeld: Leeren \\ \cline{2-2}
		& Abflugsort Destinationsfeld: Gebe "`Zurich"' ein \\ \cline{2-2}
		& Abflugsort Destinationsfeld: Warte bis die Vorschläge erscheinen \\ \cline{2-2}
		& Abflugsort Destinationsfeld: Klicke auf den ersten Vorschlag \\ \cline{2-2}
		& Ankunftsort Destinationsfeld: Leeren \\ \cline{2-2}
		& Ankunftsort Destinationsfeld: Gebe "`Palma de Mallorca"' ein \\ \cline{2-2}
		& Ankunftsort Destinationsfeld: Warte bis die Vorschläge erscheinen \\ \cline{2-2}
		& Ankunftsort Destinationsfeld: Klicke auf den ersten Vorschlag \\ \cline{2-2}
		& Zimmerangabe: Klicke auf das Feld \\ \cline{2-2}
		& Zimmerangabe: Wähle "`2 Erwachsenen"' aus \\ \cline{2-2}
		& Datumsangabe: Klicke auf das Abreisedatum \\ \cline{2-2}
		& Datumsangabe: Gehe 3 Monate in die Zukunft \\ \cline{2-2}
		& Datumsangabe: Wähle das erste Datum an \\ \cline{2-2}
		& Datumsangabe: Klicke auf das Rückreisedatum \\ \cline{2-2}
		& Datumsangabe: Wähle das dritte Datum an \\ \cline{2-2}
		& Klicke auf den Suchen Button \\ \hline
				
		\multirow{1}{*}{Flug Suchresultatseite} & Warte bis die Suchresultate dargestellt werden \\ \cline{2-2}
		& \textbf{Verifikation:} Es muss mind. ein Flug vorhanden sein. \\ \hline
	\end{tabularx} 
\end{table}

\subsection{Hotel}
\begin{table}[H] 
	\caption{Testspezifikation: Smoke Test - Hotel}
	\centering
	\rowcolors{1}{tablebodycolor}{tablerowcolor}
		
	\begin{tabularx}{0.9\textwidth}{ | l | X | } 
			\hline 
			\textbf{Seite} & \textbf{Aktion} \\ \hline 
			\multirow{1}{*}{-} & Gehe auf die Startseite \\ \hline
			\multirow{1}{*}{Startseite} & Destinationsfeld: Leeren \\ \cline{2-2}
			& Destinationsfeld: Gebe "`ber"' ein \\ \cline{2-2}
			& Destinationsfeld: Warte bis die Vorschläge erscheinen \\ \cline{2-2}
			& Destinationsfeld: Klicke auf den ersten Vorschlag \\ \cline{2-2}
			& Zimmerangabe: Klicke auf das Feld \\ \cline{2-2}
			& Zimmerangabe: Wähle "`1 Erwachsenen"' aus \\ \cline{2-2}
			& Zimmerangabe: Wähle "`1 Kind"' aus \\ \cline{2-2}
			& Zimmerangabe: Wähle für das Geburtsdatum jeweils das erste Element aus den Dropdowns aus.  \\ \cline{2-2}
			& Zimmerangabe: Füge ein Zimmer hinzu \\ \cline{2-2}
			& Zimmerangabe: Wähle "`1 Erwachsenen"' aus \\ \cline{2-2}
			& Datumsangabe: Klicke auf das Abreisedatum \\ \cline{2-2}
			& Datumsangabe: Gehe 3 Monate in die Zukunft \\ \cline{2-2}
			& Datumsangabe: Wähle das erste Datum an \\ \cline{2-2}
			& Datumsangabe: Klicke auf das Rückreisedatum \\ \cline{2-2}
			& Datumsangabe: Wähle das dritte Datum an \\ \cline{2-2}
			& Klicke auf den Suchen Button \\ \hline
					
			\multirow{1}{*}{Hotel Suchresultatseite} & Warte bis die Suchresultate dargestellt werden \\ \cline{2-2}
			& \textbf{Verifikation:} Es muss mind. ein Hotel vorhanden sein. \\ \hline
	\end{tabularx} 
\end{table}

\section{Main Tests}
Bei den Happy Paths wird überprüft, ob die Funktionalität der Webseite gemäss Spezifikation (siehe \cref{app:Funktionalitäten} \nameref{app:Funktionalitäten}) funktioniert. Die Funktionalitäten im Anhang sind priorisieriert und werden gemäss dieser umgesetzt. Begonnen wurde mit der höchsten Zahl.

\subsection{Citytrip}

\begin{table}[H] 
	\caption{Testspezifikation: Main Tests - Citytrip}
	\centering
	\rowcolors{1}{tablebodycolor}{tablerowcolor}
		
	\begin{tabularx}{0.9\textwidth}{ | l | X | } 
		\hline 
		\textbf{Seite} & \textbf{Aktion} \\ \hline 
		\multirow{1}{*}{-} & Gehe auf die Startseite \\ \hline
		\multirow{1}{*}{Startseite} & Destinationsfeld: Leeren \\ \cline{2-2}
		& Destinationsfeld: Gebe "`Amsterdam"' ein \\ \cline{2-2}
		& Destinationsfeld: Warte bis die Vorschläge erscheinen \\ \cline{2-2}
		& Destinationsfeld: Klicke auf den ersten Vorschlag \\ \cline{2-2}
		& Zimmerangabe: Klicke auf das Feld \\ \cline{2-2}
		& Zimmerangabe: Wähle "`1 Erwachsenen"' aus \\ \cline{2-2}
		& Zimmerangabe: Wähle "`1 Kind"' aus \\ \cline{2-2}
		& Zimmerangabe: Wähle für das Geburtsdatum jeweils das erste Element aus den Dropdowns aus.  \\ \cline{2-2}
		& Zimmerangabe: Füge ein Zimmer hinzu \\ \cline{2-2}
		& Zimmerangabe: Wähle "`1 Erwachsenen"' aus \\ \cline{2-2}
		& Datumsangabe: Klicke auf das Abreisedatum \\ \cline{2-2}
		& Datumsangabe: Gehe 3 Monate in die Zukunft \\ \cline{2-2}
		& Datumsangabe: Wähle das erste Datum an \\ \cline{2-2}
		& Datumsangabe: Klicke auf das Rückreisedatum \\ \cline{2-2}
		& Datumsangabe: Wähle das dritte Datum an \\ \cline{2-2}
		& Klicke auf den Suchen Button \\ \hline
		
		\multirow{1}{*}{Hotel Suchresultatseite} & Warte bis die Suchresultate dargestellt werden \\ \cline{2-2}
		& Wähle das erste Hotel aus \\ \hline
		
		\multirow{1}{*}{Hotel Konfigurationsseite} & Warte bis der "`Weiter"' Button anwählbar ist \\ \cline{2-2}
		& Klicke auf den "`Weiter"' Button \\ \hline
				
		\multirow{1}{*}{Flug Konfigurationsseite} & Warte bis die Seite geladen ist \\ \cline{2-2}
		& Warte bis der "`Weiter"' Button anwählbar ist \\ \cline{2-2}
		& Suche ein Flug welcher mehrere Hinflüge hat und wähle den zweiten aus \\ \cline{2-2}
		& Überprüfe ob sich in der Übersicht die Airline, das Datum und der Preis korrekt angepasst haben\\ \cline{2-2}
		& Suche ein Flug welcher mehrere Rückflüge hat und wähle den zweiten aus \\ \cline{2-2}
		& Überprüfe ob sich in der Übersicht die Airline, das Datum und der Preis korrekt angepasst haben\\ \cline{2-2}
	\end{tabularx} 
\end{table}

\subsection{Flight}
\begin{table}[H] 
	\caption{Testspezifikation: Main Tests - Flight}
	\centering
	\rowcolors{1}{tablebodycolor}{tablerowcolor}
		
	\begin{tabularx}{0.9\textwidth}{ | l | X | } 
		\hline 
		\textbf{Seite} & \textbf{Aktion} \\ \hline 
		\multirow{1}{*}{-} & Gehe auf die Startseite \\ \hline
		\multirow{1}{*}{Startseite} & Abflugsort Destinationsfeld: Leeren \\ \cline{2-2}
		& Abflugsort Destinationsfeld: Gebe "`new york"' ein \\ \cline{2-2}
		& Abflugsort Destinationsfeld: Warte bis die Vorschläge erscheinen \\ \cline{2-2}
		& Abflugsort Destinationsfeld: Klicke auf den ersten Vorschlag \\ \cline{2-2}
		& Ankunftsort Destinationsfeld: Leeren \\ \cline{2-2}
		& Ankunftsort Destinationsfeld: Gebe "`Zurich"' ein \\ \cline{2-2}
		& Ankunftsort Destinationsfeld: Warte bis die Vorschläge erscheinen \\ \cline{2-2}
		& Ankunftsort Destinationsfeld: Klicke auf den ersten Vorschlag \\ \cline{2-2}
		& Passagierangabe: Klicke auf das Feld \\ \cline{2-2}
		& Passagierangabe: Wähle "`2 Erwachsenen"' aus \\ \cline{2-2}
		& Passagierangabe: Wähle "`1 Kind"' aus \\ \cline{2-2}
		& Passagierangabe: Wähle für das Geburtsdatum jeweils das erste Element aus den Dropdowns aus.  \\ \cline{2-2}
		& Datumsangabe: Klicke auf das Abreisedatum \\ \cline{2-2}
		& Datumsangabe: Gehe 3 Monate in die Zukunft \\ \cline{2-2}
		& Datumsangabe: Wähle das erste Datum an \\ \cline{2-2}
		& Datumsangabe: Klicke auf das Rückreisedatum \\ \cline{2-2}
		& Datumsangabe: Wähle das dritte Datum an \\ \cline{2-2}
		& Klicke auf den Suchen Button \\ \hline
				
		\multirow{1}{*}{Flug Suchresultatseite} & Überprüfe ob beim Anpassen des Hinfluges der korrekte Flug angewählt wird: \\ \cline{2-2}
		& - Suche eine Fluggruppe mit mehreren Hinflügen \\ \cline{2-2}
		& - Speichere den ersten Flug der Fluggruppe \\ \cline{2-2}
		& - Wähle den zweiten Flug der Fluggruppe \\ \cline{2-2}
		& - Überprüfe ob der zweite Flug korrekt aktualisiert wurde \\ \cline{2-2}& Überprüfe ob beim Anpassen des Hinfluges der korrekte Flug angewählt wird: \\ \cline{2-2}
		& - Suche eine Fluggruppe mit mehreren Rückflügen \\ \cline{2-2}
		& - Speichere den ersten Flug der Fluggruppe \\ \cline{2-2}
		& - Wähle den zweiten Flug der Fluggruppe \\ \cline{2-2}
		& - Überprüfe ob der zweite Flug korrekt aktualisiert wurde \\ \cline{2-2}
		& Überprüfe ob bei allen Flügen der Totalprei angezeigt wird \\ \cline{2-2}
		& Überprüfe ob das Datum des Fluges in der Übersicht dargestellt wird. \\ \cline{2-2}
		& Überprüfe für alle Flüge: \\ \cline{2-2}
		& - Flug ist anwählbar \\ \cline{2-2}
		& - Airline Bild des Hinfluges wird angezeigt \\ \cline{2-2}
		& - Airline Bild des Rückfluges wird angezeigt \\ \cline{2-2}
		& Wähle ein Rückflug mit mehreren Hin- und Rückfluge \\ \cline{2-2}
		& Wähle den zweiten Hin- und Rückflug aus \\ \cline{2-2}
		& Speichere folgende Daten der Hin- und Rückflüge für eine spätere Überprüfung: \\ \cline{2-2}
		& - Abreisezeit des Hinfluges \\ \cline{2-2}
		& - Ankunftszeit des Hinfluges \\ \cline{2-2}
		& - Abreisezeit des Rückfluges \\ \cline{2-2}
		& - Ankunftszeit des Rückfluges \\ \cline{2-2}
		& - Preis des Fluges \\ \cline{2-2}
		& Wähle den Flug aus \\ \hline
		
		\multirow{1}{*}{Checkout: Passagierangabe} & Warte bis der Warenkorb geladen ist \\ \cline{2-2}
		& Überprüfe ob die Daten im Warenkorb mit den gewählten Daten übereinstimmen: \\ \cline{2-2}
		& - Abreisezeit des Hinfluges \\ \cline{2-2}
		& - Ankunftszeit des Hinfluges \\ \cline{2-2}
		& - Abreisezeit des Rückfluges \\ \cline{2-2}
		& - Ankunftszeit des Rückfluges \\ \cline{2-2}
		& - Preis des Fluges \\ \cline{2-2}
		& Fülle die Passagierangaben wie im \cref{sec:Konzept:Übersicht:Angaben} \nameref{sec:Konzept:Übersicht:Angaben} beschrieben ab. \\ \cline{2-2}
		& Fülle die Kundendaten wie im \cref{sec:Konzept:Übersicht:Angaben} \nameref{sec:Konzept:Übersicht:Angaben} beschrieben ab. \\ \cline{2-2}
		& Klicke auf den "`Weiter"' Button \\ \hline
		
		\multirow{1}{*}{Checkout: Bezahlart} & Warte bis die Seite geladen ist \\ \cline{2-2}
		& Warte bis der Warenkorb geladen ist \\ \cline{2-2}
		& Überprüfe ob das Kreditkarten-Formular standartmässig nicht angezeigt wird: \\ \cline{2-2}
		& - Kreditkartennummer Feld wird nicht angezeigt \\ \cline{2-2}
		& - Monats Feld wird nicht angezeigt \\ \cline{2-2}
		& - Jahr Feld wird nicht angezeigt \\ \cline{2-2}
		& - CVC Feld wird nicht angezeigt \\ \cline{2-2}
		& - Name Feld wird nicht angezeigt \\ \cline{2-2}
		& Klicke auf den Kreditkaraten Radio-Button \\ \cline{2-2}
		& Überprüfe ob das Kreditkarten-Formular angezeigt wird: \\ \cline{2-2}
		& - Kreditkartennummer Feld wird angezeigt \\ \cline{2-2}
		& - Monats Feld wird angezeigt \\ \cline{2-2}
		& - Jahr Feld wird angezeigt \\ \cline{2-2}
		& - CVC Feld wird angezeigt \\ \cline{2-2}
		& - Name Feld wird angezeigt \\ \cline{2-2}
		& Klicke auf den "`Weiter"' Button \\ \hline
		& Überprüfe ob bei jedem Kreditkarten-Formular-Feld ein Fehler angezeigt wird, da keine gültigen Kreditkarten-Daten angegeben wurden: \\ \cline{2-2}
		& - Kreditkartennummer Feld zeigt einen Fehler an \\ \cline{2-2}
		& - Monats Feld zeigt einen Fehler an \\ \cline{2-2}
		& - Jahr Feld zeigt einen Fehler an \\ \cline{2-2}
		& - CVC Feld zeigt einen Fehler an \\ \cline{2-2}
		& - Name Feld zeigt einen Fehler an \\ \cline{2-2}
		& Überprüfe Kreditkartennummer Feld Funktionalität: \\ \cline{2-2}
		& - Wenn eine 15 stellige Zahl angegeben wird erscheint ein Fehler \\ \cline{2-2}
		& - Wenn eine 16 stellige Zahl angegeben wird erscheint kein Fehler \\ \cline{2-2}
		& Überprüfe Monats Feld Funktionalität: \\ \cline{2-2}
		& - Wenn "`0"' eingegeben wird erscheint ein Fehler \\ \cline{2-2}
		& - Wenn "`8"' eingegeben wird erscheint ein Fehler \\ \cline{2-2}
		& - Wenn "`01"' eingegeben wird erscheint kein Fehler \\ \cline{2-2}
		& Überprüfe Jahr Feld Funktionalität: \\ \cline{2-2}
		& - Wenn "`a"' eingegeben wird erscheint ein Fehler \\ \cline{2-2}
		& - Wenn "`9"' eingegeben wird erscheint ein Fehler \\ \cline{2-2}
		& - Wenn "`9a"' eingegeben wird erscheint ein Fehler \\ \cline{2-2}
		& - Wenn "`a9"' eingegeben wird erscheint ein Fehler \\ \cline{2-2}
		& - Wenn "`20"' eingegeben wird erscheint kein Fehler \\ \cline{2-2}
		& Überprüfe CVC Feld Funktionalität: \\ \cline{2-2}
		& - Wenn "`a"' eingegeben wird erscheint ein Fehler \\ \cline{2-2}
		& - Wenn "`aa"' eingegeben wird erscheint ein Fehler \\ \cline{2-2}
		& - Wenn "`aaa"' eingegeben wird erscheint ein Fehler \\ \cline{2-2}
		& - Wenn "`9"' eingegeben wird erscheint ein Fehler \\ \cline{2-2}
		& - Wenn "`99"' eingegeben wird erscheint ein Fehler \\ \cline{2-2}
		& - Wenn "`424"' eingegeben wird erscheint kein Fehler \\ \cline{2-2}
		& Überprüfe ob mit korrekten Kreditkarten Daten der Checkout Prozess fortgeführt werden kann: \\ \cline{2-2}
		& - Gebe "`01234567890123456"' für die Kreditkartennummer ein \\ \cline{2-2}
		& - Gebe "`02"' für den Monat ein \\ \cline{2-2}
		& - Gebe "`21"' für das Jahr ein \\ \cline{2-2}
		& - Febe "`123"' für das CVC ein \\ \cline{2-2}
		& - Klicke auf den "`Weiter"' Button \\ \cline{2-2}
		
		
		\multirow{1}{*}{Checkout: Übersicht} & Warte bis die Seite geladen ist \\ \cline{2-2}
		& Warte bis der Warenkorb geladen ist \\ \hline
	\end{tabularx} 
\end{table}